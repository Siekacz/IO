\documentclass[12pt,a4paper]{report}
\usepackage[utf8]{inputenc}
\usepackage[OT4]{polski}
\usepackage{amsmath}
\usepackage{amsfonts}
\usepackage{amssymb}
\usepackage{fullpage}
\makeatletter
\newcommand*{\toccontents}{\@starttoc{toc}}
\makeatother
\begin{document}

\newcommand{\itab}[1]{\hspace{4em}\rlap{#1}}
\newcommand{\tab}[1]{\hspace{.2\textwidth}\rlap{#1}}

\begin{titlepage}
\begin{center}
\textsc{Studencka Pracownia Inżynierii Oprogramowania}\\[0.5cm]
\textsc{Instytut Informatyki Uniwersytetu Wrocławskiego}\\[8.3cm]

Szymon Czapiga, Bartosz Zasieczny\\[1.0cm]

\textsc{\LARGE{Dokumentacja internetowej aplikacji do wyszukiwania połączen komunikacji miejskiej i dalekobieżnej XYZ}}\\[1.0cm]

Standardy dokumentacyjne\\[8.3cm]

Wrocław, 15 października 2014\\[0.5cm]
Wersja 0.1
\end{center}
\end{titlepage}

\begin{table}[h1]
 \itab \textit{Tabela 0.} Historia zmian dokonanych w dokumencie
  \begin{center}
    \begin{tabular}{| c | c | c | c |}
    \hline
    Data & Numer Wersji & Opis & Autor \\
    \hline \hline
    2014-10-15 & 0.1 & Utworzenie dokumentu & Bartosz Zasieczny \\
    \hline
    \end{tabular}
  \end{center}
\end{table}
\textbf{\large{Spis treści}}\\[0.3cm]
\toccontents
\newpage
\section{Wstęp}
\section{Ogólny opis}
\section{Specyficzne wymagania}

\end{document}
\documentclass[12pt,a4paper]{report}
\usepackage[utf8]{inputenc}
\usepackage[OT4]{polski}
\usepackage{amsmath}
\usepackage{amsfonts}
\usepackage{amssymb}
\usepackage{fullpage}
\makeatletter
\newcommand*{\toccontents}{\@starttoc{toc}}
\makeatother
\renewcommand*{\thesection}{\arabic{section}}
\usepackage[dotinlabels]{titletoc}
\usepackage{secdot}
\sectiondot{subsection}
\newcommand{\blankpage}{
\newpage
\thispagestyle{empty}
\mbox{}
\newpage
}
\begin{document}

\newcommand{\itab}[1]{\hspace{4em}\rlap{#1}}
\newcommand{\tab}[1]{\hspace{.2\textwidth}\rlap{#1}}

\begin{titlepage}
\begin{center}
\textsc{Studencka Pracownia Inżynierii Oprogramowania}\\[0.5cm]
\textsc{Instytut Informatyki Uniwersytetu Wrocławskiego}\\[8.0cm]

Szymon Czapiga, Bartosz Zasieczny\\[1.0cm]

\LARGE{\textsc{Dokumentacja aplikacji internetowej PRZEWODNIK}}\\[8.5cm]

\begin{normalsize}

Wrocław, 20 stycznia 2015\\[0.5cm]
Wersja 1.1
\end{normalsize}
\end{center}
\end{titlepage}

\begin{table}[h1]
 \itab \textit{Tabela 0.} Historia zmian dokonanych w dokumencie
  \begin{center}
    \begin{tabular}{| c | c | c | c |}
    \hline
    Data & Numer Wersji & Opis & Autor \\
    \hline \hline
    2014-10-15 & 0.1 & Utworzenie dokumentu & Bartosz Zasieczny \\
    \hline
    2014-10-22 & 0.2 & Aktualizacja & Bartosz Zasieczny \\
    \hline
    2014-10-27 & 0.3 & Dodanie treści  & Szymon Czapiga \\
    \hline
    2014-10-28 & 0.4 & Korekta numeracji działów & Szymon Czapiga \\
    \hline
    2014-10-29 & 0.5 & Uzupełnienie treści & Szymon Czapiga, Bartosz Zasieczny \\
    \hline
    2014-12-10 & 0.6 & Uzupełnienie treści & Szymon Czapiga, Bartosz Zasieczny \\
    \hline
    2015-01-09 & 0.7 & Korekta & Szymon Czapiga\\
    \hline
    2015-01-12 & 0.8 & Uzupełnienie treści & Bartosz Zasieczny \\
    \hline
    2015-01-13 & 0.9 & Korekta & Szymon Czapiga, Bartosz Zasieczny \\
    \hline
    2015-01-20 & 1.0 & Korekta & Szymon Czapiga, Bartosz Zasieczny \\
    \hline
    2015-01-28 & 1.1 & Dodanie treści i korekta & Szymon Czapiga, Bartosz Zasieczny \\
    \hline
    \end{tabular}
  \end{center}
\end{table}
\textbf{\large{Spis treści}}\\[0.3cm]
\toccontents
\newpage
\section{Wstęp}
\subsection{Cel systemu}
	\emph{Przewodnik} wyszukuje połączenia komunikacji międzymiastowej i miejskiej, jak również umożliwia przeglądanie statystyk podróży.
\subsection{Zakres systemu}
	\emph{Przewodnik} pozwala na wyszukiwanie połączeń międzymiastowych, jak również połączeń w obrębie danego miasta. W razie potrzeby łączy te dwa typy połączeń w trasy. Obsługuje połączenia autokarowe, kolejowe oraz lotnicze (w komunikacji międzymiastowej).
\subsection{Definicje, akronimy i skróty}
	W tym podrozdziale zostaną wyjaśnione definicje, akronimy i skróty używane w dalszej części dokumentacji.
\begin{enumerate}
	\item Użytkownik -- osoba korzystająca z aplikacji.
	\item Przewoźnik -- firma oferująca usługi przewozu na danym etapie podróży.
	\item Etap podróży -- podstawowa jednostka informacji w systemie, zawierająca dane na temat bezpośredniego połączenia między dwoma adresami.
	\item Przesiadka -- czas pomiędzy dwoma etapami podróży.
	\item Trasa -- połączone ze sobą w odpowiedniej kolejności etapy podróży.
	\item Ulubione trasy -- trasy oznaczone przez użytkownika w celu szybszego wyszukiwania połączeń.
\end{enumerate}
\subsection{Uzasadnienie rynkowe zapotrzebowania na system}
	Istnieje wiele aplikacji realizujących podobne funkcje, jednak skupiają się zwykle tylko na konkretnym typie podróży -- miejskiej lub międzymiastowej. \emph{Przewodnik} łączy te dwie funkcje i udostępnia użytkownikowi kompleksową usługę wyszukiwania tras. Co więcej, użytkownik może dodawać wybrane trasy do ulubionych, co umożliwi w przyszłości szybsze wyszukiwanie. Dostępny jest również moduł ze statystykami odbytych podróży.
\newpage
\subsection{Krótki przegląd podobnych rozwiązań}
\begin{enumerate}
	\item Witryna e-podróżnik.pl -- \emph{Przewodnik} umożliwia wyszukiwanie połączeń PKS, PKP, MPK i prywatnych przewoźników autobusowych w obrębie kraju, międzynarodowych połączeń autokarowych oraz połączeń lotniczych. Jej wadą jest rozbicie funkcjonalności na różne wyszukiwarki oraz nieergonomiczny interfejs.
	\item Witryna jakdojade.pl -- wyszukuje połączenia w obrębie danego miasta.
	\item Witryna rozklad-pkp.pl -- wyszukuje połączenia kolejowe.
\end{enumerate}
\newpage
\section{Ogólny opis}
\subsection{Podstawowe funkcje}
	Podstawowymi zadaniami realizowanymi przez \emph{Przewodnika} są:
	\begin{itemize}
	 	\item wyszukiwanie połączeń,
	 	\item wyświetlanie informacji o połączeniach,
	 	\item zbieranie danych i wyświetlanie statystyk podróży.
	\end{itemize}
\subsection{Podstawowe cechy}
	\emph{Przewodnik} jest aplikacją, która ma za zadanie w jak najkrótszym czasie wskazać optymalny sposób przemieszczenia się z punktu A do punktu B, za pomocą dostępnych publicznie środków transportu. Interfejs użytkownika charakteryzuje się niewielką liczbą elementów i minimalną liczbą kroków, które trzeba wykonać, aby zdobyć informacje o poszukiwanej trasie. Prostota aplikacji pozwala, po etapie aktywnego tworzenia jej, na bardzo tanie i możliwie bezobsługowe utrzymywanie.
\subsection{Ustalenia dotyczące środowiska}
	Do działania \emph{Przewodnika} wymagana jest dowolna przeglądarka internetowa obsługująca HTML5, CSS3 oraz JavaScript. Użytkownik nie może korzystać z programów typu Adblock Plus (blokujących reklamy) i NoScript (wyłączających obsługę JavaScript w przeglądarce) -- w przeciwnym razie aplikacja wyświetli stosowny komunikat i jej używanie będzie niemożliwe.
\subsection{Relacje z innymi systemami}
	\emph{Przewodnik} do części zadań będzie wykorzystywała zewnętrzne systemy. Ten podrozdział opisuje ogólne powiązania między nimi.
\begin{itemize}
	\item dane o etapach podróży są pobierane z serwisów odpowiednich przewoźników,
	\item mapy użyte do obrazowania trasy są dostarczane przez Mapy Google,
	\item aplikacja posiada opcję połączenia z serwisami społecznościowymi, takimi jak Facebook i Google+,
	\item wykorzystanie systemu GPS do odczytywania obecnej lokalizacji.
\end{itemize}
\newpage
\subsection{Ogólne ograniczenia}
	\emph{Przewodnik}, ze względu na swoją złożoność, jak również liczbę powiązanych systemów zewnętrznych, posiada następujące ograniczenia:	
\begin{enumerate}
	\item Informacje o opóźnieniach -- dane te nie są udostępniane przez wszystkich przewoźników, w związku z tym, nie zawsze istnieje możliwość informowania o nich w \emph{Przewodniku}.
	\item Informacje o cenie biletu -- ze względu na brak danych o cenie biletu u znaczącej części przewoźników, obliczanie ceny biletu będzie opcjonalne.
	\item Informacje o dostępności miejsc -- większość przewoźników nie posiada systemu rezerwacji lub informowania o ilości wolnych miejsc.
	\item Bezpośredni zakup biletu -- część przewoźników sprzedaje bilety tylko bezpośrednio w pojeździe, przed wyjazdem.
\end{enumerate}
\subsection{Opis architektury}
\emph{Przewodnik} składa się z dwóch, niezależnych od siebie, komponentów:
\begin{enumerate}
	\item aplikacja działająca w przeglądarce internetowej.
	\item aplikacja serwerowa, stanowiąca źródło danych dla aplikacji przeglądarkowej.
\end{enumerate} 
	Aplikacja przeglądarkowa łączy się z aplikacją serwerową, będącą interfejsem do bazy danych i zewnętrznych źródeł danych. Oddzielenie aplikacji serwerowej od klienta pozwoli na łatwe rozszerzanie klienta \emph{Przewodnika} o obsługę innych platform (np. mobilnych).
\subsection{Oszacowanie pracochłonności}
\begin{enumerate}
	\item Opracowanie projektu interfejsu użytkownika -- 2 tygodnie.
	\item Opracowanie wstępnej wersji aplikacji przeglądarkowej, bez aplikacji serwerowej -- 1 miesiąc.
	\item Testy wstępnej wersji interfejsu graficznego na użytkownikach -- 2 tygodnie.
	\item Poprawki specyfikacji i implementacji interfejsu graficznego, na podstawie wniosków z testów na użytkownikach -- 1 miesiąc.
	\item Opracowanie wstępnej wersji aplikacji serwerowej -- 1 miesiąc.
	\item Testy integracyjne aplikacji serwerowej i przeglądarkowej -- 2 tygodnie.
	\item Poprawki specyfikacji i implementacji aplikacji przeglądarkowej i serwerowej, na podstawie wniosków z testów integracyjnych -- 1 miesiąc.
	\item Testy integracyjne całości systemu na użytkownikach -- 1 miesiąc.
	\item Końcowe poprawki obu aplikacji -- 2 tygodnie 
\end{enumerate}
\subsection{Oszacowanie kosztów}
\begin{enumerate}
	\item Zatrudnienie programisty HTML5/CSS/JS na okres trzech miesięcy -- 15,000 zł
	\item Zatrudnienie programisty aplikacji serwerowej na okres trzech miesięcy -- 18,000 zł
	\item Zatrudnienie grafika na okres miesiąca -- 4,000 zł
	\item Zatrudnienie specjalisty w zakresie marketingu na okres dwunastu miesięcy -- 20,000 zł
	\item Wynajęcie serwera na pierwszy rok działalności -- 2,000 zł
	\item Wykupienie kampanii reklamowej w Internecie -- 2,000 zł
	\item Usługi prawnicze -- 12,000 zł
	\item Wynajęcie firmy testującej bezpieczeństwo aplikacji -- 10,000 zł

Łącznie: 83,000 zł
\end{enumerate}
\newpage
\blankpage
\section{Specyficzne wymagania}
\subsection{Wymagania dotyczące wydajności systemu}
	\emph{Przewodnik} zwraca pierwsze wyniki w czasie mniejszym niż 5 sekund. Reszta wyników może być generowana w tle i uzupełniana w czasie kiedy użytkownik będzie już przeglądał pierwsze wyniki.

\subsection{Wymagania dotyczące zewnętrznych interfejsów}
Aplikacja działająca w przeglądarce internetowej korzysta z interfejsu programistycznego aplikacji serwerowej, dostępnego w protokole HTTP. Interfejs ten umożliwia połączenie z aplikacją serwerową dowolnemu typowi aplikacji (binarna dla komputerów PC, mobilna, internetowa), także dla aplikacji podmiotów trzecich -- w tym wypadku wymagana jest identyfikacja każdej aplikacji korzystającej z udostępnionego interfejsu (programistycznego).
\subsection{Wymagania dotyczące wykonywanych operacji}
	\emph{Przewodnik} wyszukuje połączenia pomiędzy dwoma adresami w danym mieście, jak również pomiędzy dwoma miastami. W razie potrzeby, łączy wiele różnych etapów podróży, otrzymując trasę wybraną przez użytkownika. \emph{Przewodnik} udostępnia takie informacje jak: czas podróży, godzina i miejsce przesiadek oraz dane kontaktowe przewoźnika dla każdego etapu podróży. \emph{Przewodnik} udostępnia również panel ze statystykami poprzednich podróży oraz panel opcji społecznościowych.
\newpage
\subsection{Wymagania dotyczące zasobów}
\begin{enumerate}
	\item Zespół programistów -- składa się z dwóch osób, które zajmują się pisaniem odpowiednich części systemu. Jedna zajmuje się programowaniem aplikacji internetowej i interfejsów użytkownika, używając głównie takich technologii jak HTML, CSS, JavaScript. Druga tworzy aplikację serwerową, jak również zajmuje się przeprowadzaniem testów. Zespół programistyczny może pracować niezależnie i asynchronicznie. Programiści wykonują pracę zdalnie. 
	\item Zespół testujący oprogramowanie -- za przeprowadzanie testów odpowiedzialni są wszyscy programiści, jak również zatrudniony jest, specjalnie do tego zadania, podwykonawca. Programiści mają za zadanie testować aplikację metodą białej skrzynki (każdy programista testuje element systemu, za który jest odpowiedzialny), podwykonawca zaś metodą czarnej skrzynk, np. symulując pracę użytkownika. Dzięki temu podwykonawca nie musi analizować oprogramowania przed jego testowaniem i spędzi więcej czasu rzeczywiście testując, a nie starając się zrozumieć kod.
	\item Grafik
	\item Specjalista do kontaktu z klientami i reklamy -- jego zadaniem jest przeprowadzenie kampanii reklamowej na popularnych stronach internetowych zaraz po zakończeniu prac nad \emph{Przewodnikiem}, jak również późniejsze kontynuowanie budowania marki i popularyzowania jej na portalach społecznościowych. Jego zadaniem jest również odpowiadanie na korespondencję elektroniczną nadesłaną przez użytkowników.
	\item Serwery -- w celu zapewnienia niezawodności działania usługi wymagane jest zakupienie kilku serwerów w różnych położeniach geograficznych. Wszystkie dane na serwerach są systematycznie archiwizowane. W razie awarii serwera dane nie zostaną utracone, a aplikacja w dalszym ciągu działa. Niektóre z serwerów spełniają określone zadania, takie jak przechowywanie bazy danych czy przechowywanie treści statycznej (np. grafiki). 
\end{enumerate}
\newpage
\subsection{Wymagania dotyczące sposobów testowania}
Testy podzielone są na trzy grupy:
\begin{enumerate}
	\item Automatyczne testy jednostkowe -- testowanie każdej klasy z osobna, w całkowitej separacji od innych klas. Są to testy tzw. "białej skrzynki", pisane przez dewelopera odpowiedzialnego za implementację danej klasy.
	\item Automatyczne testy integracyjne -- testy obejmujące wszystkie moduły aplikacji, przeprowadzane metodą "czarnej skrzynki", pisane przez deweloperów projektu oraz zewnętrzną firmę przeprowadzającą testy. Testy te symulują czynności użytkownika, zdefiniowane w scenariuszach testowych.
	\item Testy z użytkownikami -- testy, podczas których reprezentatywna grupa losowych osób wykonuje zadane scenariusze, następnie w krótkich formularzach wyrazi swoje opinie na temat interfejsu użytkownika i ogólnej użyteczności programu. Na tym poziomie testów będą porównywane różne wersje poszczególnych elementów interfejsu użytkownika (AB testy), pod kątem ergonomii i ilości konwersji. Testy będą przeprowadzane przez zewnętrzną firmę.
\end{enumerate}
\subsection{Wymagania dotyczące niezawodności}
\begin{enumerate}
	\item \emph{Przewodnik} jest udostępniany przez wiele serwerów, więc w razie awarii jednego z nich usługa dalej jest aktywna.
	\item Baza danych połączeń jest systematycznie aktualizowana, co zapobiega wyświetlaniu tras niemożliwych do zrealizowania. Baza danych jest również przechowywana w kilku stale aktualizowanych kopiach - pozwoli to rozłożyć obciążenie, co zmniejszy częstotliwość awarii.
\end{enumerate}
\subsection{Wymagania dotyczące etapu eksploatacji}
\begin{enumerate}
	\item Utrzymanie zespołu programistów, testerów i grafika do rozwijania i konserwacji aplikacji.
	\item Regularne testy aplikacji oraz serwerów.
	\item Przygotowywanie kampanii reklamowych.
	\item Wdrażanie nowych technologii oraz rozszerzanie funkcjonalności.
\end{enumerate}	
\subsection{Wymagania dotyczące bezpieczeństwa systemu}
	Zatrudnienie specjalnej zewnętrznej firmy do regularnego wykonywania testów bezpieczeństwa zarówno aplikacji jak i serwerów. Ważnym elementem dla bezpieczeństwa systemu jest prawne zobowiązanie pracowników do nieupubliczniania informacji o projekcie.
\end{document}
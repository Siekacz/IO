\documentclass[12pt,a4paper]{report}
\usepackage[utf8]{inputenc}
\usepackage[OT4]{polski}
\usepackage{amsmath}
\usepackage{amsfonts}
\usepackage{amssymb}
\usepackage{fullpage}
\makeatletter
\newcommand*{\toccontents}{\@starttoc{toc}}
\makeatother
\renewcommand*\thesection{\arabic{section}} % zmiana numeracji sekcji 0.X -> X
\begin{document}

\newcommand{\itab}[1]{\hspace{4em}\rlap{#1}}
\newcommand{\tab}[1]{\hspace{.2\textwidth}\rlap{#1}}

\begin{titlepage}
\begin{center}
\textsc{Studencka Pracownia Inżynierii Oprogramowania}\\[0.5cm]
\textsc{Instytut Informatyki Uniwersytetu Wrocławskiego}\\[7.3cm]

Szymon Czapiga, Bartosz Zasieczny\\[1.0cm]

\LARGE{\textsc{Dokumentacja internetowej  aplikacji PRZEWODNIK}}\\[1.0cm]

\begin{normalsize}
Standardy dokumentacyjne\\[7.0cm]

Wrocław, 15 października 2014\\[0.5cm]
Wersja 0.4
\end{normalsize}
\end{center}
\end{titlepage}

\begin{table}[h1]
 \itab \textit{Tabela 0.} Historia zmian dokonanych w dokumencie
  \begin{center}
    \begin{tabular}{| c | c | c | c |}
    \hline
    Data & Numer Wersji & Opis & Autor \\
    \hline \hline
    2014-10-15 & 0.1 & Utworzenie dokumentu & Bartosz Zasieczny \\
    \hline
    2014-10-22 & 0.2 & Aktualizacja & Bartosz Zasieczny \\
    \hline
    2014-10-27 & 0.3 & Dodanie treści  & Szymon Czapiga \\
    \hline
    2014-10-28 & 0.4 & Korekta numeracji działów & Szymon Czapiga \\
    \hline
    \end{tabular}
  \end{center}
\end{table}
\textbf{\large{Spis treści}}\\[0.3cm]
\toccontents
\newpage
\section{Wstęp}
\subsection{Cel systemu}
	Internetowa aplikacja mobilna do wyszukiwania połączeń komunikacji dalekobieżnej i miejskiej.
\subsection{Zakres systemu}
\begin{enumerate}
	\item Wyszukiwanie połączeń międzymiastowych
	\item Wyszukiwanie połączeń w obrębie danego miasta(po adresach)
	\item Łączenie połączeń międzymiastowych i miejskich w trasy “od adresu do adresu”
	\item Obsługa połączeń autokarowych, kolejowych i lotniczych w komunikacji międzymiastowej
\end{enumerate}
\subsection{Definicje, akronimy i skróty}
\begin{enumerate}
	\item Użytkownik - osoba korzystająca z aplikacji
	\item Przewoźnik - firma oferująca usługi przewozu na danym etapie podróży
	\item Etap podróży - podstawowa jednostka informacji w systemie, zawierająca dane na temat bezpośredniego połączenia między dwoma punktami
	\item Trasa - połączone ze sobą w odpowiedniej kolejności etapy podróży
\end{enumerate}
\subsection{Uzasadnienie rynkowe zapotrzebowania na system}
	Istnieje wiele aplikacji realizujących podobne funkcje jednak skupiają się zwykle tylko na konkretnym typie podróży - miejskiej bądź międzymiastowej. Proponowana aplikacja ma za zadanie połączyć te dwie funkcjonalności i udostępnić użytkownikowi kompleksową usługę wyszukiwania tras. 
\subsection{Krótki przegląd podobnych rozwiązań}
\begin{enumerate}
	\item e-podróżnik.pl - aplikacja oferuje wyszukiwanie połączeń PKS, PKP, MPK i prywatnych przewoźników autobusowych w obrębie kraju, międzynarodowych połączeń autokarowych oraz połączeń lotniczych. Jej wadą jest rozbicie funkcjonalności na różne wyszukiwarki oraz nieergonomiczny interface
	\item jakdojade.pl - wyszukuje jedynie połączenia w obrębie danego miasta
	\item rozklad-pkp.pl - wyszukuje jedynie połączenia kolejowe
\end{enumerate}
%-------------------------------	
\section{Ogólny opis}
\subsection{Podstawowe funkcje}
\begin{enumerate}
	\item Wyszukiwanie połączeń
	\item Wyświetlanie informacji o połączeniach
\end{enumerate}
\subsection{Podstawowe cechy}

\subsection{Ustalenia dotyczące środowiska}
	Przeglądarka internetowa obsługująca:
	\begin{enumerate}
		\item HTML5
		\item JavaScript
		\item CSS3
	\end{enumerate}
	Powinna mieć wyłączoną opcję blokowania reklam przez programy typu Adblock Plus.
\subsection{Relacje do innych systemów}
\begin{enumerate}
	\item Dane o etapach podróży będą pobierane z serwisów odpowiednich przewoźników
	\item Wyświetlanie danych na mapie będzie realizowane przez maps.google.com
	\item Będą istniały opcje społecznościowe dla takich serwisów jak Facebook, Google+
	\item Wykorzystanie systemu GPS do odczytywania obecnej lokalizacji
\end{enumerate}
\subsection{Ogólne ograniczenia}	
\begin{enumerate}
	\item Informacje o opóźnieniach
	\item Informacje o cenach biletów
	\item Informacje o dostępności miejsc
	\item Bezpośredni zakup biletu
\end{enumerate} 
\subsection{Opis architektury w tym model systemu (podstawowe elementy i powiązania między nimi)}
%Informacje będą zbierane z serwisów przewoźników a następnie program na serwerze będzie wyszukiwał trasy na ich podstawie 
\subsection{Oszacowanie pracochłonności}
\subsection{Oszacowanie kosztów}
%Napisanie aplikacji - ?
%Zakupienie serwera - ?
%Reklama - ?
%Usługi prawne - ?
\subsection{Harmonogram ( w postaci wykresu Gantta)}
%???
%-------------------------------
\section{Specyficzne wymagania}
\subsection{Wymagania dotyczące funkcji systemu}
\subsection{Wymagania dotyczące wydajności systemu}
\subsection{Wymagania dotyczące zewnętrznych interfejsów}
\subsection{Wymagania dotyczące wykonywanych operacji}
\subsection{Wymagania dotyczące wymaganych zasobów}
\subsection{Wymagania dotyczące sposobów weryfikacji}
\subsection{Wymagania dotyczące sposobów testowania}
\subsection{Wymagania dotyczące dokumentacji}
\subsection{Wymagania dotyczące ochrony informacji o projekcie}
\subsection{Wymagania dotyczące przenośności}
	Jedynym wymaganiem aplikacji jest posiadanie zainstalowanej przeglądarki internetowej na urządzeniu uruchamiającym aplikację.
\subsection{Wymagania dotyczące jakości}
\subsection{Wymagania dotyczące niezawodności}
\subsection{Wymagania dotyczące pielęgnacyjności}
\subsection{Wymagania dotyczące bezpieczeństwa systemu}

\end{document}
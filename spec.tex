\documentclass[12pt,a4paper]{report}
\usepackage[utf8]{inputenc}
\usepackage[OT4]{polski}
\usepackage{amsmath}
\usepackage{amsfonts}
\usepackage{amssymb}
\usepackage{fullpage}
\makeatletter
\newcommand*{\toccontents}{\@starttoc{toc}}
\makeatother
\renewcommand*\thesection{\arabic{section}} % zmiana numeracji sekcji 0.X -> X
\begin{document}

\newcommand{\itab}[1]{\hspace{4em}\rlap{#1}}
\newcommand{\tab}[1]{\hspace{.2\textwidth}\rlap{#1}}

\begin{titlepage}
\begin{center}
\textsc{Studencka Pracownia Inżynierii Oprogramowania}\\[0.5cm]
\textsc{Instytut Informatyki Uniwersytetu Wrocławskiego}\\[7.3cm]

Szymon Czapiga, Bartosz Zasieczny\\[1.0cm]

\LARGE{\textsc{Dokumentacja internetowej  aplikacji PRZEWODNIK}}\\[1.0cm]

\begin{normalsize}
Standardy dokumentacyjne\\[7.0cm]

Wrocław, 15 października 2014\\[0.5cm]
Wersja 0.4
\end{normalsize}
\end{center}
\end{titlepage}

\begin{table}[h1]
 \itab \textit{Tabela 0.} Historia zmian dokonanych w dokumencie
  \begin{center}
    \begin{tabular}{| c | c | c | c |}
    \hline
    Data & Numer Wersji & Opis & Autor \\
    \hline \hline
    2014-10-15 & 0.1 & Utworzenie dokumentu & Bartosz Zasieczny \\
    \hline
    2014-10-22 & 0.2 & Aktualizacja & Bartosz Zasieczny \\
    \hline
    2014-10-27 & 0.3 & Dodanie treści  & Szymon Czapiga \\
    \hline
    2014-10-28 & 0.4 & Korekta numeracji działów & Szymon Czapiga \\
    \hline
    \end{tabular}
  \end{center}
\end{table}
\textbf{\large{Spis treści}}\\[0.3cm]
\toccontents
\newpage
\section{Wstęp}
\subsection{Cel systemu}
	Internetowa aplikacja mobilna do wyszukiwania połączeń komunikacji dalekobieżnej i miejskiej.
\subsection{Zakres systemu}
	1. Wyszukiwanie połączeń międzymiastowych
	2. Wyszukiwanie połączeń w obrębie danego miasta(po adresach)
	3. Łączenie połączeń międzymiastowych i miejskich w trasy “od adresu do adresu”
	4. Obsługa połączeń autokarowych, kolejowych i lotniczych w komunikacji międzymiastowej
\subsection{Definicje, akronimy i skróty}
	1. Użytkownik - osoba korzystająca z aplikacji
	2. Przewoźnik 
	3. Etap podróży - podstawowa jednostka informacji w systemie, zawierająca dane nt. bezpośredniego połączenia miedzy dwoma punktami  
	4. Trasa - połączone ze sobą w odpowiedniej kolejności etapy podróży
\subsection{Uzasadnienie rynkowe zapotrzebowania na system}
	Istnieje wiele aplikacji realizujących podobne funkcje jednak skupiają się tylko na konkretnym typie podróży - miejskiej bądź międzymiastowej. Proponowana aplikacja ma za zadanie połączyć te dwie funkcjonalności i udostępnić użytkownikowi kompleksową usługę wyszukiwania tras. 
\subsection{Krótki przegląd podobnych rozwiązań}
	1. e-podróżnik - podobna funkcjonalność, okropny interface
	2. jakdojade - tylko połączenia miejskie
%-------------------------------	
\section{Ogólny opis}
\subsection{Podstawowe funkcje}
	1. Wyszukiwanie połączeń
	2. Wyświetlanie informacji o połączeniach
\subsection{Podstawowe cechy}
\subsection{Ustalenia dotyczące środowiska}
	1. Przeglądarka internetowa
\subsection{Relacje do innych systemów}
 Pobieranie baz danych z serwisów przewoźników.
\subsection{Ogólne ograniczenia}	
 Szczegółowe informacje o połączeniach
\subsection{Opis architektury w tym model systemu (podstawowe elementy i powiązania między nimi)}
%Informacje będą zbierane z serwisów przewoźników a następnie program na serwerze będzie wyszukiwał trasy na ich podstawie 
\subsection{Oszacowanie pracochłonności}
%????
\subsection{Oszacowanie kosztów}
%Napisanie aplikacji - ?
%Zakupienie serwera - ?
%Reklama - ?
%Usługi prawne - ?
\subsection{Harmonogram ( w postaci wykresu Gantta)}
%???
%-------------------------------
\section{Specyficzne wymagania}
\subsection{Wymagania dotyczące funkcji systemu}
\subsection{Wymagania dotyczące wydajności systemu}
\subsection{Wymagania dotyczące zewnętrznych interfejsów}
\subsection{Wymagania dotyczące wykonywanych operacji}
\subsection{Wymagania dotyczące wymaganych zasobów}
\subsection{Wymagania dotyczące sposobów weryfikacji}
\subsection{Wymagania dotyczące sposobów testowania}
\subsection{Wymagania dotyczące dokumentacji}
\subsection{Wymagania dotyczące ochrony informacji o projekcie}
\subsection{Wymagania dotyczące przenośności}
\subsection{Wymagania dotyczące jakości}
\subsection{Wymagania dotyczące niezawodności}
\subsection{Wymagania dotyczące pielęgnacyjności}
\subsection{Wymagania dotyczące bezpieczeństwa systemu}

\end{document}
\documentclass[12pt,a4paper]{report}
\usepackage[utf8]{inputenc}
\usepackage[OT4]{polski}
\usepackage{amsmath}
\usepackage{amsfonts}
\usepackage{amssymb}
\usepackage{fullpage}
\makeatletter
\newcommand*{\toccontents}{\@starttoc{toc}}
\makeatother
\renewcommand*\thesection{\arabic{section}} % zmiana numeracji sekcji 0.X -> X
\begin{document}

\newcommand{\itab}[1]{\hspace{4em}\rlap{#1}}
\newcommand{\tab}[1]{\hspace{.2\textwidth}\rlap{#1}}

\begin{titlepage}
\begin{center}
\textsc{Studencka Pracownia Inżynierii Oprogramowania}\\[0.5cm]
\textsc{Instytut Informatyki Uniwersytetu Wrocławskiego}\\[7.3cm]

Szymon Czapiga, Bartosz Zasieczny\\[1.0cm]

\LARGE{\textsc{Dokumentacja internetowej  aplikacji PRZEWODNIK}}\\[1.0cm]

\begin{normalsize}
Standardy dokumentacyjne\\[7.0cm]

Wrocław, 15 października 2014\\[0.5cm]
Wersja 0.5
\end{normalsize}
\end{center}
\end{titlepage}

\begin{table}[h1]
 \itab \textit{Tabela 0.} Historia zmian dokonanych w dokumencie
  \begin{center}
    \begin{tabular}{| c | c | c | c |}
    \hline
    Data & Numer Wersji & Opis & Autor \\
    \hline \hline
    2014-10-15 & 0.1 & Utworzenie dokumentu & Bartosz Zasieczny \\
    \hline
    2014-10-22 & 0.2 & Aktualizacja & Bartosz Zasieczny \\
    \hline
    2014-10-27 & 0.3 & Dodanie treści  & Szymon Czapiga \\
    \hline
    2014-10-28 & 0.4 & Korekta numeracji działów & Szymon Czapiga \\
    \hline
    2014-10-29 & 0.5 & Uzupełnienie treści & Szymon Czapiga, Bartosz Zasieczny \\
    \hline
    \end{tabular}
  \end{center}
\end{table}
\textbf{\large{Spis treści}}\\[0.3cm]
\toccontents
\newpage
\section{Wstęp}
\subsection{Cel systemu}
	Internetowa aplikacja mobilna do wyszukiwania połączeń komunikacji dalekobieżnej i miejskiej.
\subsection{Zakres systemu}
\begin{enumerate}
	\item Wyszukiwanie połączeń międzymiastowych
	\item Wyszukiwanie połączeń w obrębie danego miasta(po adresach)
	\item Łączenie połączeń międzymiastowych i miejskich w trasy
	\item Obsługa połączeń autokarowych, kolejowych i lotniczych w komunikacji międzymiastowej
\end{enumerate}
\subsection{Definicje, akronimy i skróty}
\begin{enumerate}
	\item Użytkownik - osoba korzystająca z aplikacji
	\item Przewoźnik - firma oferująca usługi przewozu na danym etapie podróży
	\item Etap podróży - podstawowa jednostka informacji w systemie, zawierająca dane na temat bezpośredniego połączenia między dwoma adresami.
	\item Przesiadka - czas pomiędzy dwoma etapami podróży
	\item Trasa - połączone ze sobą w odpowiedniej kolejności etapy podróży
\end{enumerate}
\subsection{Uzasadnienie rynkowe zapotrzebowania na system}
	Istnieje wiele aplikacji realizujących podobne funkcje jednak skupiają się zwykle tylko na konkretnym typie podróży - miejskiej bądź międzymiastowej. Proponowana aplikacja ma za zadanie połączyć te dwie funkcjonalności i udostępnić użytkownikowi kompleksową usługę wyszukiwania tras. 
\subsection{Krótki przegląd podobnych rozwiązań}
\begin{enumerate}
	\item e-podróżnik.pl - aplikacja oferuje wyszukiwanie połączeń PKS, PKP, MPK i prywatnych przewoźników autobusowych w obrębie kraju, międzynarodowych połączeń autokarowych oraz połączeń lotniczych. Jej wadą jest rozbicie funkcjonalności na różne wyszukiwarki oraz nieergonomiczny interface
	\item jakdojade.pl - wyszukuje jedynie połączenia w obrębie danego miasta
	\item rozklad-pkp.pl - wyszukuje jedynie połączenia kolejowe
\end{enumerate}
\newpage
\section{Ogólny opis}
\subsection{Podstawowe funkcje}
\begin{enumerate}
	\item  Wyszukiwanie połączeń
	\item  Wyświetlanie informacji o połączeniach
\end{enumerate}
\subsection{Podstawowe cechy}
	Aplikacja będzie:
	\begin{enumerate}
		\item Szybka
		\item Intuicyjna
		\item Ergonomiczna
		\item Przenośna
		\item Posiadała minimalistyczny interfejs
	\end{enumerate}
\subsection{Ustalenia dotyczące środowiska}
	Przeglądarka internetowa obsługująca:
	\begin{enumerate}
		\item HTM5
		\item CSS3
		\item JavaScript
	\end{enumerate}
	Użytkownik nie może korzystać z programów typu Adblock Plus(blokujących reklamy) i NoScript(wyłączających obsługę JavaScript w przeglądarce)
\subsection{Relacje do innych systemów}
\begin{enumerate}
	\item Dane o etapach podróży będą pobierane z serwisów odpowiednich przewoźników
	\item Wyświetlanie danych na mapie będzie realizowane przez maps.google.com
	\item Będą istniały opcje społecznościowe dla takich serwisów jak Facebook, Google+
	\item Wykorzystanie systemu GPS do odczytywania obecnej lokalizacji
\end{enumerate}
\subsection{Ogólne ograniczenia}	
\begin{enumerate}
	\item Informacje o opóźnieniach
	\item Informacje o cenie biletu
	\item Informacje o dostępności miejsc
	\item Bezpośredni zakup biletu
\end{enumerate}
\subsection{Opis architektury w tym model systemu (podstawowe elementy i powiązania między nimi)}
Aplikacja będzie składać się z dwóch, niezależnych od siebie, komponentów:
\begin{itemize}
	\item Aplikacja działająca w przeglądarce internetowej.
	\item Aplikacja serwerowa, stanowiąca źródło danych dla aplikacji przeglądarkowej.
\end{itemize} 
	Aplikacja przeglądarkowa będzie łączyć się z aplikacją serwerową, która będzie stanowić interfejs do bazy danych i zewnętrznych źródeł danych.
\subsection{Oszacowanie pracochłonności}
\begin{itemize}
	\item Opracowanie projektu interfejsu użytkownika - 2 tygodnie.
	\item Opracowanie wstępnej wersji aplikacji przeglądarkowej, bez aplikacji serwerowej - 1 miesiąc.
	\item Testy wstępnej wersji interfejsu graficznego na użytkownikach - 2 tygodnie.
	\item Poprawki specyfikacji i implementacji interfejsu graficznego, na podstawie wniosków z testów na użytkownikach - miesiąc.
	\item Opracowanie wstępnej wersji aplikacji serwerowej - 1 miesiąc.
	\item Testy integracyjne aplikacji serwerowej i przeglądarkowej - 2 tygodnie.
	\item Poprawki specyfikacji i implementacji aplkacji przeglądarkowej i serwerowej, na podstawie wniosków z testów integracyjnych - 1 miesiąc.
	\item Testy integracyjne całości systemu na użytkownikach - 1 miesiąc.
	\item Końcowe poprawki obu aplikacji - 2 tygodnie 
\end{itemize}
\subsection{Oszacowanie kosztów}
\begin{enumerate}
	\item Zatrudnienie programisty HTML5/CSS/JS na okres 3 miesięcy - 15,000 zł
	\item Zatrudnienie programisty aplikacji serwerowej na okres 3-ch miesięcy - 18,000 zł
	\item Zatrudnienie grafika na okres 1-go miesiąca - 4,000 zł
	\item Zatrudnienie specjalisty d/s marketingu na okres 12-tu miesięcy - 20,000 zł
	\item Wynajęcie serwera na 1-szy rok działaności - 2,000 zł
	\item Wykupienie kampanii reklamowej w Internecie - 2,000 zł
	\item Usługi prawnicze - 12,000 zł
	\item Wynajęcie firmy testującej bezpieczeństwo aplikacji - 10,000 zł

Łącznie: 83,000 zł
\end{enumerate}
%\subsection{Harmonogram ( w postaci wykresu Gantta)}
\newpage
\section{Specyficzne wymagania}
%\subsection{Wymagania dotyczące funkcji systemu}
\subsection{Wymagania dotyczące wydajności systemu}
	Aplikacja ma zacząć zwracać pierwsze wyniki w czasie mniejszym niż 5 sekund. Reszta wyników może być generowana w tle i uzupełniana w czasie kiedy użytkownik będzie już przeglądał pierwsze wyniki.
\subsection{Wymagania dotyczące zewnętrznych interfejsów}
Aplikacja działająca w przeglądarce internetowej korzystać będzie z interfejsu programistycznego aplikacji serwerowej, dostępnego przez protokół HTTP. Interfejs ten będzie umożliwiać połączenie się z aplikacją serwerową dowolnemu typowi aplikacji (natywna dla komputerów PC, mobilna, internetowa), także dla aplikacji podmiotów trzecich - w tym wypadku wymagana będzie identyfikacja każdej aplikacji korzystającej z udostępnionego interfejsu programistycznego.
\subsection{Wymagania dotyczące wykonywanych operacji}
	Aplikacja powinna:
	\begin{enumerate}	
		\item Wyszukiwać połączenia pomiędzy dwoma adresami w danym mieście
		\item Wyszukiwać połączenia pomiędzy dwoma miastami
		\item W razie potrzeby łączyć wiele różnych etapów podróży w trasę
		\item Informować o czasie podróży
		\item Informować o przesiadkach
		\item Przekazywać dane kontaktowe przewoźnika dla każdego etapu podróży 
	\end{enumerate}
\subsection{Wymagania dotyczące wymaganych zasobów}
\begin{enumerate}
	\item Zespół programistów
	\item Zespół testerów
	\item Grafik
	\item Specjalista do kontaktu z klientami
	\item Serwery
\end{enumerate}
%\subsection{Wymagania dotyczące sposobów weryfikacji}
\subsection{Wymagania dotyczące sposobów testowania}
Testy będą podzielone na trzy grupy:
\begin{itemize}
	\item Automatyczne testy jednostkowe
	\item Automatyczne testy integracyjne
	\item Testy na użytkownikach
\end{itemize}
%\subsection{Wymagania dotyczące dokumentacji}
\subsection{Wymagania dotyczące ochrony informacji o projekcie}
	Prawne zobowiązania pracowników do nieupubliczniania informacji o projekcie.
\subsection{Wymagania dotyczące przenośności}
	Aplikacja może być uruchamiana na każdym urządzeniu spełniającym wymagania z sekcji "Ustalenia dotyczące środowiska".
%\subsection{Wymagania dotyczące jakości}
\subsection{Wymagania dotyczące niezawodności}
\begin{enumerate}
	\item Aplikacja będzie udostępniana przez wiele serwerów, więc w razie awarii jednego z nich dalej będzie aktywna.
	\item Baza danych połączeń będzie systematycznie aktualizowana co zapobiegnie wyświetlaniu tras niemożliwych do zrealizowania.
\end{enumerate}
\subsection{Wymagania dotyczące pielęgnacyjności}
\begin{enumerate}
	\item Utrzymanie zespołu programistów, testerów i grafika do rozwijania i konserwacji aplikacji
	\item Regularne testy aplikacji oraz serwerów
	\item Przygotowywanie kampanii reklamowych
	\item Wdrażanie nowych technologii oraz rozszerzanie funkcjonalności
\end{enumerate}	
\subsection{Wymagania dotyczące bezpieczeństwa systemu}
	Zatrudnienie specjalnej zewnętrznej firmy do regularnego przeprowadzania testów bezpieczeństwa zarówno aplikacji jak i serwerów.
\end{document}